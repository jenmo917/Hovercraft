\subsection{Hårdvara}

Lista för att starta svävaren:
\begin{enumerate}
\item Se till batterierna är laddade och inkopplade.
\item Slå på de båda strömbrytarna på sidan av chassit (se till att batterierna är kopplade till dessa och att klämmorna sitter fast)
\item Slå på brytaren på alla strömförsörjningskorten. Alla statusdioder ska lysa starkt och klart. På grund av backspänning kan dioderna lysa trots att kortet är avstäng. Brytarna på strömförsörjningskorten står med fördel påslagna hela tiden och endast chassi brytarna används.
\item Om telefonen är inkopplad via USB ska den nu få ström från svävaren.
\item Svävaren är nu igång och mottaglig till uppkoppling från fjärrkontrollen\\
\end{enumerate}


Kort felsökning om det inte funkar:
\begin{itemize}
\item Undersök sladdar, sladd från batteri till chassi strömbrytaren kan ibland sitta löst
\item Strömbrytare påslagna? (alla statusdioder ska lysa stark och klart. På grund av backspänning kan dioderna lysa trots att kortet är avstängt)
\item Får inte telefonen kontakt med ADK? Lyser dioder på ADK, om inte se till  att 5 V kontakt är ansluten till ADK och strömförsörjning påslagen. Om påslaget och dioder lyser testa att  trycka på reset knappen på ADK. Fortfarande inget? Dra ur USB-kontakt till telefon. 
\item Går lyftfläktarna ojämnt? Kolla spänningsnivå ut från de strömförsörjningskort som är kopplade till fläktarna och se om de är lika annars skruva lite på potentiometern för att ändra.
\item Går inte framdrivningsfläktarna? Kolla att alla strömförsörjningskort är påslagna, H-bryggorna behöver 5 V, 12 V och en justerbar spänning för drivning av fläktarna.
\end{itemize}
