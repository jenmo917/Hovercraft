\subsection{H-brygga}
För att styra hastighet och riktning av framdrivningsmotorerna designades en
h-brygga för att kunna leverera 12 V 3 A. Design och tester för h-bryggan
arbetades fram parallellt med ett projekt i en annan kurs (TNE089 -
Elektromagnetisk kompatibilitet och mönsterkortdesign) för en grundlig design.
För en mer utförlig rapport om h-bryggans design finns rapporten från TNE089 i
appendix \ref{apx:H-bridge}.

\subsubsection{Resultat}
Ett kretsschema designades med Altium Designer 6, kretsschemat kan ses i figur \ref{fig:h_brygga_schema}.

\begin{landscape}
\begin{figure}[htbp!]
\centering
\includegraphics[width=18cm]{../../includes/figures/h_brygga_schematic}
\caption{Kretsschema.}
\label{fig:h_brygga_schema}
\end{figure}
\end{landscape}

PCB-layouten kan ses i figur \ref{fig:pcb_layout}, enbart hålmonterade komponenter användes för att förenkla monteringen då utrymme på svävaren ej var ett problem.

\begin{figure}[htbp!]
\centering
\includegraphics[width=11cm]{../../includes/figures/H_brygga_pcb}
\caption{PCB-layout för h-bryggan.}
\label{fig:pcb_layout}
\end{figure}

Efter att en protoyp av h-bryggan tagits fram i institutionens PCB-labb och
designen verifierats beställdes mönsterkorten från ITead Studio \cite{ITead Studio}. Den färdiga h-bryggan kan ses i figur \ref{fig:mounted_h_bridge}.

\begin{figure}[htbp!]
\centering
\includegraphics[width=10cm]{../../includes/figures/Hbridge}
\caption{Den färdiga h-bryggan.}
\label{fig:mounted_h_bridge}
\end{figure}

\subsubsection{Diskussion}
H-bryggan fungerade bra och det finns även möjlighet till att styra mer
effektkrävande motorer då MOSFET:arna är specade för upp till 55~V 110~A ifall
att en mer kraftfull motor skulle användas.

\subsubsection{Ekonomi}
En tabell över komponenter till en h-brygga ses i tabell \ref{tbl:BOM h-brygga}.
Den totala kostnaden för två h-bryggors komponenter var 703,60~SEK. Korten
beställdes från ITead Studio vilket kostade 396,50~SEK. Totalt kostade
båda h-bryggorna 1100,10~SEK.

\begin{table}[htbp!]
\centering
\caption{Tabell över komponenter}
\label{tbl:BOM h-brygga}
\begin{tabular}{|l|l|r|c|c|}
\hline
\textbf{Komponent} & \textbf{Information} &
\textbf{Schemanot.} & \textbf{St} & \textbf{SEK/st} \\
\hline
MOSFET driver & HIP4081AIPZ & U1 & 1 & 102 \\
\hline
Schottky diode & SB340 & D1-D5 & 6 & 4.93\\
\hline
Power MOSFET & IRF3205PBF & Q1-Q4 & 4 & 43.5\\
\hline
Small signal transistor & 2N3906 & Q5 & 1 & 0.26\\
\hline
Resistor & 100~k$\Omega$ & R9, R10 & 2 & 0.43\\
\hline
Resistor & 10~k$\Omega$ & R5-R8 & 4 & 0.35\\
\hline
Resistor & 100~$\Omega$ & R1-R4 & 4 & 0.27\\
\hline
Resistor & 16~k$\Omega$ & R11 & 1 & 0.31\\
\hline
Resistor & 3.3~k$\Omega$ & R12 & 1 & 0.21\\
\hline
Quad 2-input NAND & 7400 & U2, U3 & 2 & 4.31\\
\hline
Electrolytic capacitor & 1~uF & C1, C2 & 2 & 1.68\\
\hline
Ceramic capacitor & 1~uF & C3 & 1 & 4.31\\
\hline
Male Connector 90$^{\circ}$ & 2~p & P1-P4 & 4 & 0\\
\hline
PCB Terminal Block &  4~p & P5 & 1 & 7.60\\
\hline
Fuse holder & 6.7 A & F1 & 1 & 18.21\\
\hline
\end{tabular}	
\end{table}