\subsection{Chassi}
Vid design av chassit användes CAD-programmet SolidWorks 2012 för att göra
3D-modeller över chassits olika delar. En prototyp modellerades i SolidWorks och
tillverkades för att användas till olika tester innan det slutliga chassit
skulle designas och tillverkas.

\subsubsection{Material}
Chassit är gjort av ett lätt skivmaterial som består av två tunna “pappskivor”
med skumplast emellan. Skivorna är 3,1 mm tjocka och är lätta att klippa, skära
och limma. Skivorna skänktes till projektgruppen av Björn-Åke
Sköld för att göra en prototyp av svävaren. Materialet användes till prototypen
men även till den slutliga svävaren.

Dysorna till fläktarna är gjorda av gjutrör av papp  och träramen som håller
ihop höljet som döljer elektroniken är av furu. Till kjolen användes ett
ripstop-tyg som är ett slitstarkt tunt tyg som används bl.a. till drakar.

\subsubsection{Prototyp}
Arbetet inleddes med att göra en CAD-modell över prototypen. Den gjordes dubbelt
så stor jämfört med den svävare som en tidigare projektgrupp har byggt.
Björn-Åke trodde nämligen att man skulle kunna göra en lite större svävare men
som ändå är lättare än den gamla vid användning av hans material.

Prototypen blev 800x570 mm och 100 mm hög. Tanken var att den gamla svävaren
skulle fästas ovanpå prototypen för att se om den kunde lyfta och sväva då. En annan tanke var
att även kjolen skulle ha testats på prototypen för att sedan kunna flyttas över
till den slutliga svävaren. Eftersom det inte skulle bli helt lätt att fästa den
gamla svävaren på prototypen testades prototypen direkt med de lyftfläktar som
skulle användas. Den kunde då lyfta och hållas svävande trots att ingen kjol
användes. Kjolen designades som en variant av fingerkjol bestående av flera
segment men hann inte sys innan en idé på en ny design dök upp.

\subsubsection{Ny design}
Efter ett möte med Björn-Åke, då han fick se prototypen och CAD-modellen av
kjolen, bestämdes att både chassit och kjolen skulle konstrueras om. Detta för
att kjolsegmenten blev för små att sy och att kjolen i sin helhet inte skulle
kunna hålla kvar luften så att ett lufttryck kunde erhållas.

I den nya designen
kunde höjden halveras på svävaren och bredden ökas 3 cm, detta för att göra
sävaren stabilare. Det nya chassit har en bagkjol istället
eftersom det är lättare att sy en sådan kjol och den kan dessutom hålla kvar
luften på ett bättre sätt så ett lufttryck erhålls. En bagkjol är inte lika
flexibel som en fingerkjol när det gäller att ta sig över hinder men i detta
projekt är bagkjolen tillräckligt flexibel.

Kjolen sitter fast i en plattform med kardborreband och eftersom det försvinner
mycket luft vid kardborrefästena har inte några hål gjorts i kjolen. På så vis
erhålls även ett lufttryck i kjolen som gör att svävaren kan bära en större
totalvikt.

Plattformen som kjolen är fäst runt är 800x600 mm och består av två
plattor av det lätta skivmaterialet. Den undre plattan är lite mindre än den
övre och de sitter ihop med distanser för att få en luftspalt på 10 mm mellan
plattorna.

En träram konstruerades för att kunna fästa fläktar, motorer och elektronik
i. Den sitter fast i plattformen med kardborreband och håller även ihop höljet
som döljer elektroniken med kardborreband.

Svävaren är uppbyggd på ett modulärt sätt och de olika delarna är placerade på
ett sätt som ger låg tyngdpunkt. Den är även uppbyggd symmetriskt med en jämn
viktfördelning över hela svävarens bottenyta. En 3D-modell över svävaren visas i
figur \ref{fig:CAD_Hover}.

\begin{figure}[htbp!] 
\centering 
\includegraphics[width=8cm]{../includes/figures/CAD_Hovercraft.png} 
\caption{3D-modell över svävaren.} 
\label{fig:CAD_Hover} 
\end{figure}

\subsubsection{Vidareutveckling}
I en vidareutveckling av svävaren skulle materialet kunna bytas ut mot ett mer
hållbart och vattentåligt material. Skivmaterialet har en tendens att slitas
sönder när delarna ska tas isär vid kardborrefästena.

Träramen som är av furuskulle kunna bytas ut till ett lättare träslag som
balsaträ eller ett annat lätt material.

Genom att skala upp modellen kan en större svävare göras med samma design. Men
det som behöver tänkas på då är att de fläktar och motorer som väljs klarar av
den nya vikten.

\subsubsection{Ekonomi}
I tabell \ref{tbl:Kostnad chassi} redovisas kostnaderna för chassit.
\begin{table}[htbp!]
\centering
\caption{Kostnader för chassi.}
\label{tbl:Kostnad chassi}
\begin{tabular}{l|l}
\hline
Pris & Pris [SEK] \\
\hline
Kardborre & 147 \\
Ripstoptyg & 186 \\
Ramvirke & 61 \\
Motorfästen & 58 \\
Skruv & 29 \\
Mobilhållare & 147 \\
\hline
Summa & 628

\end{tabular}
\end{table}

