\subsection{ADK}
Ett Android Development Kit (ADK) användes för att kontrollera svävaren. 
Ett ADK är ett utvecklingskort speciellt anpassat för att användas tillsammans med Android.
Det fanns några alternativ men gruppen valde att införskaffa två olika kort för att testa 
vilket som till slut skulle användas på svävaren. De två korten var Embedded Artists kort 
``Android Open Accessory Application Kit'' (AOAA) \cite{AOAA} och Arduinos ``Arduino ADK'' \cite{Arduino ADK}. 

\subsubsection{Resultat}
Det bestämdes att kortet från Arduino skulle användas då de som skulle
arbeta med denna delen av mjukvaruutvecklingen tyckte att kortet var enklare att
programmera för än Embedded Artists motsvarighet och fortfarande uppfyllde de
krav som hade på styrning av svävarelektroniken.

\subsubsection{Diskussion}
Det skulle gå att byta utvecklingskort helt och köra på Embedded Artists AOAA vilket öppnar upp för fler funktioner, 
t.ex kommunikation via CAN-bussen. Detta var dock inget krav som ställdes på
kommunikationen i detta projekt.

\subsubsection{Ekonomi}
Då gruppen valde att införskaffa två olika utvecklingskort gick kostnaden upp något. Arduino ADK kostade 57.71 \euro~och 
Embedded Artists AOAA kostade 65.18 \euro~som vid införskaffandets växlingskurs
uppgick till totalt 1385 SEK.
