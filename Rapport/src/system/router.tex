\subsection{Router}
Routern var tänkt att användas för två ändamål:
\begin{itemize}
\item sensorplattform för reglersystemet
\item och kommunikationslänk mellan olika system på fjärrkontrollen,
routerplattformen och ADKn.
\end{itemize}

Som nämdes tidigare så finns möjligheten att använda privata
Androidplattformar som fjärkontroll. Men eftersom att svävaren i sig beror
på vilka sensorer som finns tillgängliga, samt att minimera förändringar på
svävaren, så beslutades det att köpa in en dedikerad plattform. Olika modeller
av Androidplattformar undersöktes för att hitta ett lämpligt objekt att inhandla.

\subsubsection{Resultat}
Efter efterforskningar beslutade projektgruppen att köpa in en Samsung Google
Nexus S, då denna ansågs ha de sensorer som krävdes för att implementera ett väl
fungerande reglersystem. Den var även dokumenterat bra att använda tillsammans
med en ADK \cite[Supported Android Devices]{Android ADK Arduino}.
Androidversionen som ligger på som standard på telefonen kommer även direkt
från Google, detta gör det enkelt att uppdatera till nyare versioner av
operativsystemet. Den var även ett av de billigare alternativen.

\subsubsection{Ekonomi}
Kostnaden för inköpet av telefonen var 1400 SEK. 

\subsubsection{Diskussion}
Vid inköp av ny Androidplattform så rekommenderar projektgruppen att köpa en som
innehåller ett gyro. Detta är väldigt användbart vid utveckling av reglersystem
för positionering- och hastighetskontroll.
