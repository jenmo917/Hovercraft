\subsection{Fjärrkontroll}
Som nämnt i avsnitt \ref{subsec:system/fjarrkontroll} så har Fjärrkontrollen tre
stora uppgifter: generera styrsignaler till svävaren, kommunicera med de andra
enheterna samt att kunna logga händelser på svävaren.

Applikationen är uppdelad tre huvudsakliga delar som var och en utgörs av en
Android-service. Dessa delar implementerar funktionalitet för kommunikation,
loggning av sensordata samt generering av styrsignaler. Applikationen innehåller
även Android-activity som fungerar som inställningsmeny för fjärrkontrollen.

Genomgång på hur kommunikationen mellan fjärrkontrollen och de andra enheterna
sker gås igenom i avsnitt \ref{subsec:commlink} och generering av
styrsignaler behandlas i \ref{subsec:styr och regler}. Nedan beskrivs hur
loggning fungerar.

\subsubsection{Loggning av sensordata}
Funktionaliteten för att logga olika typer av sensordata sker genom att kryssa
i vilka sensorer som data ska sparas i från. Insamlade data sparas som textfiler
på fjärrkontrollens SD-minne.
På den senaste versionen av applikationen finns funktionalitet för loggning av
data från fjärrkontrollens accelerometer, svävarens accelerometer samt från
svävarens ultraljudssensorer. Vilken data som skall loggas ställs in i
inställningsmenyn. Om data från svävaren önskas skickas ett kommando till den
som gör att den svarar med ett paket innehållande önskad data.

\subsubsection{Resultat}
Fjärrkontrollen möter de krav som ställs i kravspecifikationen. Vid skrivandet
av rapporten hade tester dock bara genomförts på plattformar av
smartphone-modell.

\begin {itemize}
\item Räckvidd 20 meter
\item Indikera när kommunikationen med svävaren bryts.
\item Kunna kommunicera med svävaren. Samt ge de kommandon som gränssnittet till
svävaren tillåter.
\end {itemize}

\subsubsection{Vidareutveckling}
Som brukligt så blir man aldrig riktigt klar med utveckling av mjukvara, det
finns alltid saker att finslipa. Loggningsfunktionen bör kontrolleras noggrant
och utökas så att fler sensorer går att logga. Protocol Buffer är inte
implementerat som protokoll och det skulle med relativ enkelhet kunna ändras.
